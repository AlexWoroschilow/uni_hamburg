\documentclass[11pt,a4paper]{article}
\usepackage{indentfirst}
\usepackage[hmargin=2cm,vmargin=3cm,bindingoffset=0.5cm]{geometry}
\usepackage{amssymb}
\usepackage{amsmath}
\usepackage{setspace}
\usepackage{lipsum}

\begin{document}

%\noindent GSA
% \vspace{0.66cm}
\noindent \textbf{Aufgabe 3} \\

\vspace{0.66cm}



\noindent Gegeben $ \delta $ Kostenfunktion, so dass f\"{u}r alle Editoperationen $ \alpha \to \beta $ die folgenden Eigenschaften gelten: \\
\indent \indent $ \delta (\alpha \to \beta) = \delta(\beta \to \alpha) $ \\
\indent \indent $ \delta(\alpha \to \beta) = 0 \Leftrightarrow \alpha = \beta $ \\


\noindent 1.

\doublespace
\noindent Seien $ u, v \in A* $ wobei $ \alpha_1\alpha_2...\alpha_h = u und \beta_1\beta_2...\beta_h = v. $ \\
Annahme: $ edist_\delta(u,v) \ne edist_\delta(v,u). $ \\

Dann von der Definition, \\
\indent \indent $ min\{\delta(A*) | A* $ Alignment von u und v $ \} \ne min\{\delta(A*') | A*' $ Alignment von v und u \} \\
\indent \indent wobei $ A* = (\alpha_1 \to \beta_1, ... , \alpha_h \to \beta_h) $ 
und $ A*' = (\beta_1 \to \alpha_1, ... ,\beta_h \to \alpha+h). $ \\

Dann $ \delta(A*) = \sum_{i=1}^{h}  \delta(\alpha_i \to \beta_i) $ \\
\indent \indent und $  \delta(A*') = \sum_{i=1}^{h}  \delta(\beta_i \to \alpha_i) $ \\
\indent \indent \indent \indent \indent $ = \sum_{i=1}^{h}  \delta(\alpha_i \to \beta_i) $ von der Definition der Kostenfunktion \\
\indent \indent \indent \indent \indent $ = \delta(A*) $ \\
\indent \indent $ \Rightarrow min\{\delta(A*')\} = min\{\delta(A*)\} $ \\
\indent \indent $ \Rightarrow $ Widerspruch. \\

Deshalb $ edist_\delta(u,v) = edist_\delta(v,u). $ \\
 \par{\raggedright\texttt{}}
\hfill $  \Box $\\



\noindent 2. \\
Seien $ u,v \in A* $\\
$\underline \Rightarrow | $

Sei $ edist_\delta(u,v) = 0. $ 

Annahme: $ u \ne v $ 

\indent \indent $ \Rightarrow \delta(u \to v) \ne 0, \delta(v \to u) \ne 0 $

\indent \indent $ \Rightarrow \delta(A*) = \sum_{i=1}^{h}  \delta(\alpha_i \to \beta_i) $

\indent \indent $ \Rightarrow min\{\delta(A*) | A* $ Alignment von $ u $ und $ v \} > 0 $

\indent \indent $ \Rightarrow edist_\delta(u,v) \ne 0 $

\indent \indent $ \Rightarrow $ Widerspruch.

Deshalb $ u = v. $ \\



$ \underline \Leftarrow | $

Sei $ u = v. $

Dann $ \delta(u \to v) = \delta(v \to u) = 0 $ von der Definition der Kostenfunktion.

Dann die Definition von $ E_\delta \Rightarrow E_\delta(i,j) = 0 ,  \forall i,j $

\indent \indent $ \Rightarrow E(u,v) = 0 $

\indent \indent $ \Rightarrow edist_\delta(u,v) = 0. $

 \par{\raggedright\texttt{}}
\hfill $  \Box $\\
	

	\end{document}
	
